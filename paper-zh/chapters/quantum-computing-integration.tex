\chapter{量子计算集成用于Alpha挖掘}
\label{chap:quantum}

\section{概述}

量子计算为探索Alpha表达式的组合空间提供了指数级加速潜力。本章讨论量子算法在Alpha挖掘系统中的集成,重点关注实际实施策略和预期收益。

\section{用于Alpha发现的量子算法}

\subsection{量子近似优化算法(QAOA)}

QAOA可以用于搜索最优Alpha表达式:

\begin{lstlisting}[language=Python, caption=QAOA用于Alpha搜索]
class QuantumAlphaSearcher:
    def __init__(self, num_qubits: int = 20, num_layers: int = 3):
        self.num_qubits = num_qubits
        self.num_layers = num_layers
        self.cost_hamiltonian = None
        self.mixing_hamiltonian = None
        
    def encode_alpha_space(self, operators: List[str], fields: List[str]):
        """Encode alpha expression space into qubits"""
        # Each qubit represents presence/absence of operator or field
        # Total qubits = len(operators) + len(fields)
        self.operator_qubits = len(operators)
        self.field_qubits = len(fields)
        self.total_qubits = self.operator_qubits + self.field_qubits
        
    def build_cost_hamiltonian(self, historical_performance: dict):
        """Build cost Hamiltonian from historical alpha performance"""
        # Cost function: -Sharpe - Fitness + Turnover + Penalty
        def cost_function(bitstring: str) -> float:
            # Decode bitstring to alpha expression
            alpha_expr = self.decode_bitstring(bitstring)
            
            # Get historical performance if available
            if alpha_expr in historical_performance:
                perf = historical_performance[alpha_expr]
                cost = -(perf['sharpe'] * 0.5 + perf['fitness'] * 0.3) + \
                       perf['turnover'] * 0.2
            else:
                # Penalty for unknown expressions
                cost = 1.0
            
            return cost
        
        # Build Hamiltonian matrix
        size = 2 ** self.total_qubits
        H = np.zeros((size, size))
        
        for i in range(size):
            bitstring = format(i, f'0{self.total_qubits}b')
            cost = cost_function(bitstring)
            H[i, i] = cost
        
        self.cost_hamiltonian = H
        return H
    
    def qaoa_circuit(self, gamma: List[float], beta: List[float]):
        """Construct QAOA circuit"""
        # Initialize in uniform superposition
        state = np.ones(2 ** self.total_qubits) / np.sqrt(2 ** self.total_qubits)
        
        # Apply QAOA layers
        for layer in range(self.num_layers):
            # Apply cost Hamiltonian
            state = self.apply_cost_hamiltonian(state, gamma[layer])
            
            # Apply mixing Hamiltonian
            state = self.apply_mixing_hamiltonian(state, beta[layer])
        
        return state
    
    def optimize_qaoa(self, historical_performance: dict):
        """Optimize QAOA parameters"""
        self.build_cost_hamiltonian(historical_performance)
        
        # Initialize parameters
        gamma = np.random.uniform(0, np.pi, self.num_layers)
        beta = np.random.uniform(0, np.pi, self.num_layers)
        
        # Classical optimization
        from scipy.optimize import minimize
        
        def objective(params):
            gamma = params[:self.num_layers]
            beta = params[self.num_layers:]
            state = self.qaoa_circuit(gamma, beta)
            
            # Expected cost
            cost = np.real(np.dot(state.conj(), 
                                 np.dot(self.cost_hamiltonian, state)))
            return cost
        
        result = minimize(objective, 
                         np.concatenate([gamma, beta]),
                         method='COBYLA')
        
        optimal_gamma = result.x[:self.num_layers]
        optimal_beta = result.x[self.num_layers:]
        
        return optimal_gamma, optimal_beta
    
    def sample_candidates(self, gamma: List[float], beta: List[float], 
                         num_samples: int = 100) -> List[str]:
        """Sample candidate alpha expressions from QAOA state"""
        state = self.qaoa_circuit(gamma, beta)
        probabilities = np.abs(state) ** 2
        
        # Sample according to probabilities
        samples = np.random.choice(2 ** self.total_qubits, 
                                 size=num_samples, 
                                 p=probabilities)
        
        candidates = []
        for sample in samples:
            bitstring = format(sample, f'0{self.total_qubits}b')
            alpha_expr = self.decode_bitstring(bitstring)
            candidates.append(alpha_expr)
        
        return candidates
\end{lstlisting}

\subsection{变分量子本征求解器(VQE)}

VQE用于寻找最优Alpha组合:

\begin{lstlisting}[language=Python, caption=VQE实现]
class VQEAlphaOptimizer:
    def __init__(self, ansatz_depth: int = 3):
        self.ansatz_depth = ansatz_depth
        self.hamiltonian = None
        
    def build_ansatz(self, num_qubits: int, params: np.ndarray):
        """Build hardware-efficient ansatz"""
        # Initialize |0...0>
        state = np.zeros(2 ** num_qubits)
        state[0] = 1.0
        
        param_idx = 0
        
        for layer in range(self.ansatz_depth):
            # Rotation gates
            for qubit in range(num_qubits):
                theta = params[param_idx]
                state = self.apply_ry(state, qubit, theta)
                param_idx += 1
            
            # Entangling gates
            for qubit in range(num_qubits - 1):
                state = self.apply_cnot(state, qubit, qubit + 1)
        
        return state
    
    def optimize_alpha_combination(self, alpha_pool: List[AlphaResult]):
        """Find optimal combination of alphas using VQE"""
        # Build Hamiltonian from alpha correlations
        num_alphas = len(alpha_pool)
        self.hamiltonian = self.build_correlation_hamiltonian(alpha_pool)
        
        # Initialize parameters
        num_params = self.ansatz_depth * num_alphas
        params = np.random.uniform(0, 2 * np.pi, num_params)
        
        # Optimize
        from scipy.optimize import minimize
        
        def objective(params):
            state = self.build_ansatz(num_alphas, params)
            energy = np.real(np.dot(state.conj(), 
                                   np.dot(self.hamiltonian, state)))
            return energy
        
        result = minimize(objective, params, method='L-BFGS-B')
        
        # Get optimal state
        optimal_state = self.build_ansatz(num_alphas, result.x)
        
        # Extract alpha weights
        weights = np.abs(optimal_state) ** 2
        
        return weights
\end{lstlisting}

\section{混合量子-经典框架}

\subsection{集成架构}

\begin{figure}[h]
\centering
\begin{tikzpicture}[
    node distance=1.5cm,
    quantum/.style={rectangle, draw=purple!50, fill=purple!10, thick, minimum width=2.5cm, minimum height=1cm, text centered, rounded corners},
    classical/.style={rectangle, draw=blue!50, fill=blue!10, thick, minimum width=2.5cm, minimum height=1cm, text centered, rounded corners},
    data/.style={ellipse, draw=orange!50, fill=orange!10, thick, minimum width=2cm, minimum height=0.8cm, text centered},
    arrow/.style={->, >=stealth, thick},
    qarrow/.style={->, >=stealth, thick, purple!70, dashed}
]
    % Quantum layer
    \node[quantum] (qaoa) {QAOA\\Search};
    \node[quantum, right=of qaoa] (vqe) {VQE\\Optimization};
    \node[quantum, below=of qaoa] (encode) {Quantum\\Encoder};
    
    % Classical layer
    \node[classical, below=of encode, yshift=-0.5cm] (eval) {Classical\\Evaluator};
    \node[classical, right=of eval] (optim) {Parameter\\Optimizer};
    \node[classical, below=of eval, yshift=-0.5cm] (wqapi) {WorldQuant\\Brain API};
    
    % Data
    \node[data, left=of qaoa] (history) {Historical\\Performance};
    \node[data, right=of vqe, xshift=-1.5cm] (candidates) {Candidate\\Alphas};
    
    % Arrows
    \draw[arrow] (history) -> (qaoa);
    \draw[qarrow] (qaoa) -> (encode);
    \draw[qarrow] (vqe) -> (encode);
    \draw[arrow] (encode) -> (eval);
    \draw[arrow] (eval) -> (optim);
    \draw[arrow] (optim) -> (wqapi);
    \draw[arrow] (wqapi) -> (eval);
    \draw[arrow] (eval) -> (candidates);
    \draw[arrow, dashed, bend right=30] (candidates) to node[above, sloped] {Feedback} (history);
    
    % Labels
    \node[above=0.3cm of qaoa, font=\small] {Quantum Layer};
    \node[below=0.3cm of wqapi, font=\small] {Classical Layer};
\end{tikzpicture}
\caption{混合量子-经典架构}
\label{fig:quantum-hybrid}
\end{figure}

\subsection{实现}

\begin{lstlisting}[language=Python, caption=混合系统集成]
class QuantumEnhancedGenerator(EnhancedTemplateGeneratorV3):
    """Generation Two with Quantum Enhancement"""
    
    def __init__(self, *args, **kwargs):
        super().__init__(*args, **kwargs)
        
        # Quantum components
        self.quantum_searcher = QuantumAlphaSearcher(num_qubits=20)
        self.vqe_optimizer = VQEAlphaOptimizer()
        self.use_quantum = True  # Toggle quantum enhancement
        
    def quantum_enhanced_search(self, region: str, num_candidates: int = 50):
        """Use quantum search to find promising alpha expressions"""
        if not self.use_quantum:
            return self.classical_search(region, num_candidates)
        
        # Get historical performance
        historical = self.get_historical_performance(region)
        
        # Encode alpha space
        operators = self.load_operators()
        fields = self.get_data_fields(region, 
                                     self.region_configs[region].universe, 
                                     self.region_configs[region].delay)
        self.quantum_searcher.encode_alpha_space(operators, fields)
        
        # Optimize QAOA
        gamma, beta = self.quantum_searcher.optimize_qaoa(historical)
        
        # Sample candidates
        candidates = self.quantum_searcher.sample_candidates(
            gamma, beta, num_candidates
        )
        
        # Validate and filter
        valid_candidates = [
            c for c in candidates 
            if self.validate_template_syntax(c, fields)
        ]
        
        return valid_candidates
    
    def quantum_alpha_portfolio_optimization(self, alpha_pool: List[AlphaResult]):
        """Optimize portfolio of alphas using VQE"""
        # Find optimal weights
        weights = self.vqe_optimizer.optimize_alpha_combination(alpha_pool)
        
        # Build weighted portfolio
        portfolio = []
        for i, alpha in enumerate(alpha_pool):
            if weights[i] > 0.01:  # Threshold
                portfolio.append({
                    'alpha': alpha,
                    'weight': weights[i]
                })
        
        return portfolio
\end{lstlisting}

\section{预期量子优势}

\subsection{搜索空间探索}

量子叠加允许同时探索 $2^n$ 个状态:

\begin{equation}
|\Psi\rangle = \frac{1}{\sqrt{2^n}} \sum_{x=0}^{2^n-1} |x\rangle
\end{equation}

对于 $n=20$ 个量子比特,这代表 $2^{20} = 1,048,576$ 次同时搜索。

\begin{figure}[h]
\centering
\begin{tikzpicture}[
    scale=0.9,
    qubit/.style={circle, draw=blue!50, fill=blue!10, thick, minimum size=0.8cm},
    state/.style={rectangle, draw=green!50, fill=green!10, thick, minimum width=1.5cm, minimum height=0.6cm, text centered, font=\tiny},
    arrow/.style={->, >=stealth, thick},
    label/.style={font=\small}
]
    % Qubits
    \node[qubit] (q1) at (0,3) {$q_1$};
    \node[qubit] (q2) at (1.5,3) {$q_2$};
    \node[qubit] (q3) at (3,3) {$q_3$};
    \node[label, below=0.3cm of q2] {Qubits};
    
    % Superposition states
    \node[state] (s0) at (0,1) {$|000\rangle$};
    \node[state] (s1) at (1.5,1) {$|001\rangle$};
    \node[state] (s2) at (3,1) {$|010\rangle$};
    \node[state] (s3) at (4.5,1) {$|011\rangle$};
    \node[state] (s4) at (6,1) {$|100\rangle$};
    \node[state] (s5) at (7.5,1) {$|101\rangle$};
    \node[state] (s6) at (9,1) {$|110\rangle$};
    \node[state] (s7) at (10.5,1) {$|111\rangle$};
    
    % Arrows showing superposition
    \draw[arrow, blue!70] (q1) -> (s0);
    \draw[arrow, blue!70] (q1) -> (s4);
    \draw[arrow, blue!70] (q2) -> (s0);
    \draw[arrow, blue!70] (q2) -> (s2);
    \draw[arrow, blue!70] (q3) -> (s0);
    \draw[arrow, blue!70] (q3) -> (s1);
    
    % All states connected (superposition)
    \draw[arrow, dashed, green!70] (s0) to[bend left=20] (s1);
    \draw[arrow, dashed, green!70] (s1) to[bend left=20] (s2);
    \draw[arrow, dashed, green!70] (s2) to[bend left=20] (s3);
    \draw[arrow, dashed, green!70] (s3) to[bend left=20] (s4);
    \draw[arrow, dashed, green!70] (s4) to[bend left=20] (s5);
    \draw[arrow, dashed, green!70] (s5) to[bend left=20] (s6);
    \draw[arrow, dashed, green!70] (s6) to[bend left=20] (s7);
    
    % Label
    \node[label, below=0.3cm of s4] {All $2^3 = 8$ states simultaneously};
    
    % Classical comparison
    \node[label, right=1cm of q3] {Classical: 1 state at a time};
    \node[label, right=1cm of s7] {Quantum: All states at once};
\end{tikzpicture}
\caption{量子叠加:同时状态探索}
\label{fig:quantum-superposition}
\end{figure}

\subsection{纠缠用于相关性}

量子纠缠可以捕获算子之间的相关性:

\begin{equation}
|\psi_{entangled}\rangle = \frac{1}{\sqrt{2}}(|00\rangle + |11\rangle)
\end{equation}

这使得能够发现经典方法可能遗漏的相关算子模式。

\begin{figure}[h]
\centering
\begin{tikzpicture}[
    qubit/.style={circle, draw=blue!50, fill=blue!10, thick, minimum size=1cm},
    entangled/.style={circle, draw=purple!50, fill=purple!20, thick, minimum size=1.2cm},
    arrow/.style={->, >=stealth, thick, purple!70},
    label/.style={font=\small}
]
    % Entangled qubits
    \node[entangled] (q1) at (0,2) {$q_1$};
    \node[entangled] (q2) at (3,2) {$q_2$};
    
    % Correlation line
    \draw[arrow, purple!70, line width=2pt] (q1) to[bend up=30] node[above] {Entangled} (q2);
    \draw[arrow, purple!70, line width=2pt] (q2) to[bend down=30] (q1);
    
    % States
    \node[label, below=0.5cm of q1] {$|0\rangle$ or $|1\rangle$};
    \node[label, below=0.5cm of q2] {$|0\rangle$ or $|1\rangle$};
    
    % Correlation explanation
    \node[label, below=1.5cm of q1, text width=6cm, text centered] {
        If $q_1 = |0\rangle$ then $q_2 = |0\rangle$\\
        If $q_1 = |1\rangle$ then $q_2 = |1\rangle$\\
        \textbf{Perfect correlation}
    };
    
    % Application
    \node[label, above=0.5cm of q1, text width=6cm, text centered] {
        Operator correlations:\\
        $ts\_rank$ often pairs with $ts\_delta$
    };
\end{tikzpicture}
\caption{量子纠缠用于算子相关性发现}
\label{fig:quantum-entanglement}
\end{figure}

\subsection{干涉用于放大}

量子干涉放大有前景的区域:

\begin{equation}
P(x) = |\langle x | \psi \rangle|^2
\end{equation}

相长干涉增加了采样高质量Alpha的概率。

\section{实际考虑}

\subsection{当前限制}

\begin{itemize}
    \item \textbf{噪声}:当前量子硬件(NISQ)存在显著噪声
    \item \textbf{经典模拟}:完整模拟限制在约30个量子比特
    \item \textbf{编码效率}:高效编码Alpha空间具有挑战性
    \item \textbf{混合开销}:经典-量子接口增加开销
\end{itemize}

\subsection{实施策略}

\begin{enumerate}
    \item \textbf{从经典模拟开始}:经典地使用量子算法
    \item \textbf{小问题实例}:在减少的算子/字段集上测试
    \item \textbf{混合方法}:结合量子和经典方法
    \item \textbf{逐步迁移}:随着硬件可用性提高,迁移到真实硬件
\end{enumerate}

\section{性能预测}

\subsection{预期改进}

\begin{table}[h]
\centering
\begin{tabular}{lcc}
\toprule
指标 & 经典方法 & 量子增强(预测) \\
\midrule
搜索空间覆盖 & $10^6$ / 小时 & $10^{12}$ / 小时 \\
发现率 & 8-12\% & 15-25\% \\
相关性发现 & 有限 & 增强 \\
计算时间 & 小时 & 分钟(对于大空间) \\
\bottomrule
\end{tabular}
\caption{预测的量子优势}
\end{table}

\subsection{量子优势何时出现}

量子优势预期在以下情况下出现:
\begin{itemize}
    \item 搜索空间 $> 10^{10}$ 组合
    \item 算子之间存在强相关性
    \item 需要同时探索
    \item 真实量子硬件可用(50+量子比特,低错误率)
\end{itemize}

\section{总结}

量子计算集成提供:
\begin{itemize}
    \item 指数级搜索空间探索
    \item 通过纠缠发现相关性
    \item 放大有前景的区域
    \item 投资组合优化能力
\end{itemize}

虽然当前硬件限制需要经典模拟,但该框架设计用于无缝过渡到真实量子硬件,随着技术成熟,提供指数级加速的路径。
